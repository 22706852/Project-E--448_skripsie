\graphicspath{{testingandcal/fig/}}

\chapter{Testing and Calibrations}
\label{chap:testing}

\section{Overview}
\label{sec:testingoverview}

To test and display the capabilities of the device an application, various test where done when walking on certain surfaces. An important thing to consider is that the shoe used is a running shoe and running shoes have heels that are lifted to make a heel strike more comfortable. This can be seen in all test that the heel sensors experience more pressure than the rest of the sensors. The temperature was not taken into consideration as FSR cells are not sensitive to temperature conditions. Please see figure \ref{fig:footsensordevice} in appendix \ref{appen:footweardevice} to understand the legends of the graphs to following in this section.

\section{Calibration}
\label{sec:calibration}

The foot sensor calibration was done by the previous student  by using a compression testing machine, the Instron 3345. This machine can apply a compression force up to 40kN according to the previous student. The machine was set up to apply a certain amount of force and slowly release. The force was applied to a single cell and the output ADC where recorded for this cell. The student also stated in his report that due to the machine having a slow release all ADC values between the interval $F_N - 1 < F_N < F_N + 1$, where $F_N$ is applied force, had been recorded and an average ADC value was calculated. The same test was done for the cells that are connected to the ADS1115. The results were recorded and plotted as seen in figure \ref{fig:calibration}
\clearpage
\begin{figure}[!htb]
    \centering
    \includegraphics[width = 0.8\linewidth]{calibration.png}
    \caption{Sensor calibration graph}
    \label{fig:calibration}
\end{figure}

\section{Bare foot}

By standing bare foot on the pressure we can see what a mid foot standing patterns should look like. This will be used as a control as it will become apparent that the shoes has quite the influence on how the device behaves.
\begin{figure}[!htb]
    \centering
    \includegraphics[width = 0.55\linewidth, height = 8cm]{barefeet.jpg}
    \caption{Bare foot mid-stance test}
    \label{fig:calibration}
\end{figure}
\section{Basic strike patterns}
\label{sec:basicstrike}
\begin{figure}[!htb]
    \centering
    \includegraphics[width = 0.6\linewidth]{frontsfoot.jpg}
    \caption{Fore foot push-off with shoes}
    \label{fig:frontstrike}
\end{figure}
\begin{figure}[!htb]
    \centering
    \includegraphics[width = 0.6\linewidth]{midfoot.jpg}
    \caption{Mid-stance with shoes}
    \label{fig:midstrike}
\end{figure}
\clearpage
\begin{figure}[!htb]
    \centering
    \includegraphics[width = 0.6\linewidth]{heelstrike.jpg}
    \caption{Heel-strike with shoes}
    \label{fig:hellstrike}
\end{figure}
\clearpage
\section{Different surfaces}
\label{sec:surfaces}

The following test was conducted from a standing start and then, at a steady rate, walking across two different surfaces. When comparing the following two graphs, we can see how the grass causes a damping effect. The damping effect makes sense as the grass is soft and absorbs some force applied to the FSR cells  

\begin{figure}[!htb]
    \centering
    \includegraphics[width = 1.1\linewidth]{flatsurface.png}
    \caption{Graph that shows the force applied to each FSR cell when walking on a flat surface}
    \label{fig:hardflat}
\end{figure}

\begin{figure}[!htb]
    \centering
    \includegraphics[width = 1.1\linewidth]{grass_surface.png}
    \caption{Graph that shows the force applied to each FSR cell when walking on a grass/soft surface}
    \label{fig:softsurf}
\end{figure}
\clearpage
\section{Inclines}
\label{sec:inclines}
The following test was conducted from a standing start and then walking up and down a hill or incline. The surface  of the incline was hard and somewhat uniform. Here it is seen that there is much less focus on the heel part of the foot when walking up and down. It is also clear that much shorter and more frequent steps was taken when walking downhill. 

\begin{figure}[!htb]
    \centering
    \includegraphics[width = 1.1\linewidth]{uphill.png}
    \caption{Graph that shows the force applied to each FSR cell when walking up a hill/incline}
    \label{fig:uphill}
\end{figure}

\begin{figure}[!htb]
    \centering
    \includegraphics[width = 1.1\linewidth]{downhill.png}
    \caption{Graph that shows the force applied to each FSR cell when walking down a hill/incline}
    \label{fig:downhill}
\end{figure}

\clearpage
\section{Stairs}
The following test was conducted from a standing start and then walking up and down stairs. The stairs had a small surface area and therefore when walking on them only a part of the foot would make contact. When walking up the stairs it is apparent that only the front area pf the made contact with the ground. It is also interesting that one can see the last step was at the top of the staircase as for the last step shows the heel area had more applied force.
\label{sec:stairs}
\begin{figure}[!htb]
    \centering
    \includegraphics[width = 1.1\linewidth]{upstairs.png}
    \caption{Graph that shows the force applied to each FSR cell when walking upstairs}
    \label{fig:upstairs}
\end{figure}
\begin{figure}[!htb]
    \centering
    \includegraphics[width = 1.1\linewidth]{downstairs.png}
    \caption{Graph that shows the force applied to each FSR cell when walking downstairs}
    \label{fig:downstairs}
\end{figure}