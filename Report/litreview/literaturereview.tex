\graphicspath{{litreview/fig/}}

\chapter{Literature Review}
\label{chap:litreview}
\section{Foot strike patterns and Gait analysis}
\label{sec:Gait}
Running is a popular everyday physical activity across the world. Many statics proves this\cite{statistaresearchdepartment2020}. Although running is a simple activity it involves complex movements an integration of muscles, joints and varies body parts. This leads to running being a common physical activity that commonly cause injuries. In depth studies on foot strike patterns elsewhere \cite{doi:10.2519/jospt.2015.6019}, \cite{CAVANAGH1980397
}, \cite{matheuso.almeidaptphdirenes.davisptphdalexandred.lopesptphd2015}, \cite{lauram.andersondanielr.bonannoharvif.hart&christianj.barton2020} review the basics of foot strike patterns and the biomechanics of running. This is beyond the scope of this technical report, but these reviews do provide relevant information to better understand what cause s these injuries. These studies do indicate that considering the foot and foot motion during walking and running is very crucial to understanding why running can commonly lead to injuries. It has been proven in studies such as \cite{kennethp.clarklaurencej.ryanpeterg.weyand2014} that running performance, energy requirements and musculoskeletal stresses are directly related to foot strike patterns and action-reaction force between the limb and the ground. Many factors like foot strike patterns, footwear conditions, running speed and environment conditions can have many effects on the biomechanics of a human during running. According to \cite{matheuso.almeidaptphdirenes.davisptphdalexandred.lopesptphd2015} there are three primary foot strike patterns namely: forefoot, rearfoot (heel strike) and midfoot. Forefoot  striking is when the antrerior region of the foot strikes the ground first. Midfoot striking is when the posterior and anterior parts of the foot hit the ground simultaneously, and rearfoot or heel strike is when the heel or posterior area of the foot strikes the ground initially. According to \cite{marEfootstrike} and \cite{matheuso.almeidaptphdirenes.davisptphdalexandred.lopesptphd2015} there is a high prevalence of heel strikers for both mid-distance and ling-distance runners. This would explain why running shoes are heavily padded at the heel part of the shoe. This makes the landing proses more comfortable.
\begin{figure}[!htb]
    \centering
    \includegraphics[width = 0.7\linewidth]{Screenshot 2022-10-11 153343.png}
    \caption{Image found in an online article\cite{mass4d2017} illustrating the Foot Strike Patterns in Runners.}
    \label{fig:footstrike}
\end{figure}



\newpage
\section{Technology used for Gait analysis}

\section{Bluetooth Low Energy BLE}
\label{sec:ble}

\subsection{How Does BLE Work?}
\label{sec:howdoesblework}

When using Bluetooth Low Energy it is important to know the roles of each device. In all BLE application there are two roles which are the \textbf{central} and the \textbf{peripheral} devices. The peripheral device will be the device that broadcasts or advertises information and the central device will be scanning for information. A good visual representation of how BLE works is to think of an advertising board where the peripheral device keeps pinning new info onto the board and the central device scans the board and uses the information available. These two devices have their own unique addresses. The peripheral will be advertising information to any near devices while the central will for any device or devices that are advertising information. When the central device finds the advertised information a connection attempt is made. Once a connection is established, the central device can start read and writing information from and to the peripheral device.
\begin{figure}[!h]
    \centering
    \includegraphics[width = 0.6\linewidth]{BLE.png}
    \caption{Basic BLE overview}
    \label{fig:bleoverview}
\end{figure}


\subsection{Services, Characteristics and Descriptor}
\label{sec:servicesandcharacteristics}
 The information transfer system for BLE can be seen in the following hierarchical order. A service is a collection of characteristics and each characteristic has a descriptor that describes the characteristic. See the basic illustration below \ref{fig:ble_roles}.

\begin{figure}[!h]
    \centering
    \includegraphics[width = 0.5\linewidth]{blehierarchy.png}
    \caption{Basic hierarchy of BLE}
    \label{fig:ble_roles}
\end{figure}

Each \textbf{service} has its own unique identifying code called a UUID. This is to allow one peripheral device to have multiple services. This code can be 128-bit long for each service. A service can also be seen as a group of capabilities. For example, a smartwatch that can measure hart rate, temperature and track your GPS location. These three capabilities can be grouped under one service and can be called the activity service. This method of grouping information allows the central device to better understand and use the information that the peripheral is advertising.

The capabilities mentioned in the above example are better known as \textbf{characteristics}. Each characteristic has its own unique identifying code called a UUID. This is to allow one service to have multiple characteristics. A characteristic can be seen as a single capability. For example, the characteristic of measuring heart rate. This characteristic can be seen as a single capability of the activity service. 

Each characteristic can have a \textbf{descriptor} that describes the characteristic. A descriptor can be seen as a single piece of information about the characteristic. For example, the descriptor of the characteristic of measuring heart rate can be the range of the heart rate measurement. This descriptor can be seen as a single piece of information about the characteristic of measuring heart rate.


\section{Force Sensitive Resistors}
A Force Sensitive Resistor (FSR) is a material which changes resistance when a force is
applied. A FSR typically consists of 3 layers; a top resistive polymer layer, a bottom thin
film polymer layer with conductive traces and a middle spacer layer separating the top
and bottom layers. The three layers are often enclosed in a flexible polymer.
When there is no pressure applied the top resistive layer and bottom conductive layer
are completely separated by the spacer and act as an open circuit (infinite resistor).
As pressure is applied to the FSR the top resistive layer is pressed against the bottom
conductive layer causing the resistance to decrease.

\section{Open GLES for android}
\label{sec:OpenGL}
OpenGL is an open source, graphics library for high performance 2D and 3D graphics rendering. OpenGL|ES is flavor of OpenGL specifically intended for embedded and mobile devices. OpenGL|ES is a cross-plaftorm high performance graphics API that can be used by Android devices. Android supports both the framework API and the Native Development Kit (NDK) of OpenGL.
