\graphicspath{{appendices/fig/}}
\chapter{Project Planning Schedule}
\makeatletter\@mkboth{}{Appendix}\makeatother
\label{appen:projectplan}
    \begin{tabular}{ | m{5em} |c|m{12cm}| }
      \hline
      \textbf{Date} & \textbf{Week} & \textbf{Description} \\
      \hline
      Jul 18-23 & 1 & Project planning and scope \\ 
      \hline
      25-30 & 2 & Study prototype devices and how it can be used.\\ 
      \hline
      Aug 01-6 & 3 & Literature study on the topic at hand. This is everything that is correlated to running and foot strike patterns \\ 
      \hline
      8-13 and 15-20 & 4 and 5 & Start developing the Arduino code for the device. Try to get basic ADC readings from the IEEE foot sensor The previous student could not provide previously written code. \\ 
      \hline
      22-27 & 6 & Reach BLE and how it can be used to transmit data from an Arduino device. Start developing the code as soon as I have good understanding of BLE for Arduino \\ 
      \hline
      Aug 29 - Sept 3& 7 & Test Week \\ 
      \hline
      5-10 & 8 & Recess \\ 
      \hline
      12-17 and 19-24  & 9 and 10 & Start building the Android application and understand BLE regarding Android devices. Display the data transmitted from the Arduino in a text view \\ 
      \hline
      26- 1 Oct and 3-8 & 11 and 12 & Build a heatmap view using OpenGL|ES. This requires a good understanding of OpenGL|ES and custom Android views  \\ 
      \hline
      10-15 & 13 & Use the previous students calibrations to build another custom view that displays the force exerted in newton on each cell \\ 
      \hline
      17-22  & 14 & Test the application. This includes using the device on difference surfaces and walking up or down a hill. See if temperature has an effect on the cell readings \\ 
      \hline
      24-29 & 15 & Present application to a person who has knowledge of foot strike patterns to see if an application of such would be of use in the industry and reflect on findings \\ 
      \hline
      31 October 2022 & 16 & Report Deadline. Prepare for oral examination \\ 
      \hline
    \end{tabular}
\chapter{Outcomes Compliance}
\makeatletter\@mkboth{}{Appendix}\makeatother
\label{appen:derivations_bigramseg}

\begin{center}.
  \begin{tabular}{ | m{4cm} |c|m{10cm}| }
    \hline
    \textbf{ELO} & \textbf{Chapter} & \textbf{Description} \\
    \hline
    1. Problem solving & 1,2,3,4 & Chapter 1 demonstrates the identification and formulation of an engineering problem.
    
    Chapter 2 and 3 demonstrates the analysis of the problem and previous solutions
    
    Chapter 4 demonstrates how the problem was solved by developing an android application\\ 
    \hline
    2. Application of scientific and engineering knowledge & 3,4,5 & Chapter 3 and 4 demonstrates the use of mathematics and engineering fundamentals to design and program an application that can do advance calculations.\\ 
    \hline
    3. Engineering Design & 3,4 & Chapter 3 demonstrates the understanding of engineering design. Chapter 4 demonstrates the design and synthesis of and engineering idea by developing Arduino and Java code.   \\ 
    \hline
    4. Investigations, experiments and data analysis & 5 & Chapter5 demonstrates the investigations of different ways to test the device an application. The data could be retrieved from the various tests and the data was analyzed. \\ 
    \hline
    5. Engineering methods, skills, and tools, including Information Technology & 4,Appendix C & Chapter 4 demonstrates the use of programming skills in C++ and Java. In Chapter 4 mathematical tools such as Desmos was used. Appendix C demonstrates the design and print of a 3D model using ZBrush\\ 
    \hline
    6. Professional and technical communications& 6 & Chapter 6 demonstrates that the device an application were demonstrated to people in the field of sports science. Their feedback can be used for future work and recommendations. \\ 
    \hline
    7. Individual Work & All & The whole of the project was conducted individually \\ 
    \hline
    8. Independent learning ability& 2,4 & Chapter 2 demonstrates the ability to learn new concepts and skills and chapter 4 demonstrates how these newly found concepts and skills were implemented.\\
    \hline
  \end{tabular}
  \end{center}

\chapter{3D Model of device casing}
\makeatletter\@mkboth{}{Appendix}\makeatother
\label{appen:3dmodel}

\begin{figure}[!htb]
  \centering
  \includegraphics[width = 1\linewidth]{ortos_1.jpg}
  \label{fig:CustomTableLayout}
\end{figure}

\begin{figure}[!htb]
  \centering
  \includegraphics[width = 1\linewidth]{ortos_2.jpg}
  \label{fig:CustomTableLayout}
\end{figure}

\begin{figure}[!htb]
  \centering
  \includegraphics[width = 1\linewidth]{ortos_complete.jpg}
  \label{fig:CustomTableLayout}
\end{figure}

\chapter{Prototype Device}
\makeatletter\@mkboth{}{Appendix}\makeatother
\label{appen:prototype}

\begin{figure}[!htb]
  \centering
  \includegraphics[width = 0.7\linewidth]{device3.jpg}
  \label{fig:CustomTableLayout}
\end{figure}

\begin{figure}[!htb]
  \centering
  \includegraphics[width = 0.7\linewidth]{device1.jpg}
  \label{fig:CustomTableLayout}
\end{figure}

\begin{figure}[!htb]
  \centering
  \includegraphics[width = 0.7\linewidth]{device2.jpg}
  \label{fig:CustomTableLayout}
\end{figure}

\chapter{Smart Footwear Sensing Solutions by IEE}
\makeatletter\@mkboth{}{Appendix}\makeatother
\label{appen:footweardevice}

\begin{figure}[!htb]
  \centering
  \includegraphics[width = 0.7\linewidth]{footsensor.png}
  \caption{IEE foot sensor specifications and labeling for each cell}
  \label{fig:footsensordevice}
\end{figure}
